\documentclass{article}
\usepackage{amsmath}
\usepackage{amssymb}
% \usepackage{fullpage}
\usepackage{enumerate}
\usepackage{xcolor}
\renewcommand{\P}{\mathbf{P}}
\newcommand{\NP}{\mathbf{NP}}
\newcommand{\EXP}{\mathbf{EXP}}
\newcommand{\NEXP}{\mathbf{NEXP}}
\newcommand{\PH}{\mathbf{PH}}
\newcommand{\PSPACE}{\mathbf{PSPACE}}
\newcommand{\TIME}{\mathbf{TIME}}
\newcommand{\NTIME}{\mathbf{NTIME}}
\newcommand{\E}{\mathbf{E}}

\newtheorem{proposition}{Prop}

\begin{document}
	\begin{center}
		\Large
		6.541/18.405 Problem Set 0
		
		\vspace{3pt}
		\normalsize
		due on February 15, 11:59pm
	\end{center}
	
	% {\bf Rules:} You may discuss homework problems with other students and you may work in groups, but we require that you {\em try to solve the problems by yourself before discussing them with others}. Think about all the problems on your own and do your best to solve them, before starting a collaboration. If you work in a group, include the names of the other people in the group in your written solution. {\bf Write up your {\em own} solution to every problem}; don't copy answers from another student or any other source. You should be able to explain and write your solutions all by yourself. Cite {\bf all} references that you use in a solution (books, papers, people, etc) at the end of each solution. 
	
	% We encourage you to use \LaTeX, to compose your solutions. The source of this file is also available on Piazza, to get you started!
	
	% {\bf How to submit:} Use Gradescope entry code 2P3PEN
		
	\section*{Problem 1 (2 points)} 
	\subsection*{Question}
    For a Turing machine $M$, let $\langle M \rangle$ denote the description of $M$ in binary.
    Prove that the following language is undecidable: 
	\begin{multline*}
		L = \{ \langle M\rangle \mid \text{for all $n$ and input strings $x$ of length $n$, } \\
		\text{$M$ is a Turing machine that halts on $x$ in $18n^3+405$ steps} \}
	\end{multline*}

	\subsection*{Answer}

	I will reduce this to the halting problem.

	To solve the halting problem, given $\bar{M}$, we must determine whether $\bar{M}$ halts on the empty tape.

	Let $M$ be the Turing machine which does the following.  On input $x$, it first simulates $\bar{M}$ for $|x|$ steps.  If $\bar{M}$ has halted, $M$ then runs for $|x|^4$ more steps, and then halts.  Otherwise, if $\bar{M}$ has not halted, $M$ halts.

	For contradiction, suppose $L$ is decidable.  Let $D$ be a Turing machine that decides it.
	
	Say that $D$ accepts $M$.  Then $M$ halts within $18|x|^3 + 405$ steps on every input $x$.  Noting that $18 \times 100^3 + 405 < 100^4$, this means $M$ does not run for $|x|^4$ or more steps on any $x$ of length greater than 100.	Thus, it must be that for all $n > 100$, $\bar{M}$ does not halt within $n$ steps.  Thus $\bar{M}$ must not halt.

	Conversely, say $D$ does not accept $M$.  Then there is some $x$ such that $M$ runs for more than $18|x|^3 + 405$ steps.  This can only happen if $\bar{M}$ halts within $|x|$ steps.  Thus $\bar{M}$ halts.

	Thus, if $L$ is decidable, so is the halting problem.

    \section*{Problem 2 (2 points)}

	\subsection*{Question}

    Prove that if $\text{\sc 3Sat} \in P$, then there is a polynomial-time algorithm for solving $\text{\sc Search-3Sat}$ (as defined in lecture 1). That is, under the hypothesis, there is a polynomial-time algorithm that given any 3CNF formula $\phi$, the algorithm outputs a satisfying assignment to $\phi$ (when one exists) in polynomial time.
 
	\subsection*{Answer}

	% Note that in this problem I use $1$ to represent truth and $-1$ to represent falsity, so I can use multiplication notation for negation.

	Say $\text{\sc 3Sat} \in P$, and let $M$ be a Turing machine that solves $\text{\sc 3Sat}$ in polynomial time.  The following polynomial time algorithm can then be used to solve $\text{\sc Search-3Sat}$.

	Let $\phi$ be a given 3CNF formula, with variables $x_1, \dots, x_n$.
	% and clauses $c_1, \dots, c_m$, where $c_m = s_{a, m} x_{a_ m} \wedge s_{b, m} x_{b_m} \wedge s_{c, m} x_{c_m}$ where the $s \in \{-1, +1\}$ indicate which variables are negated.

	First, call $M$ on $\phi$.  If it rejects, return that no solution exists to $\phi$.

	Next, if $n = 1$, evaluate $\phi$ on $x_1 = 1$.  If it is true, return $x_1 = 1$; else, return $x_1 = 0$.

	Now say $n > 1$.  Construct the CNF formula $\phi_1$by setting $x_1$ to $1$ in $\phi$.  Call $M$ on $\phi_1$.  If it accepts, assign $x_1 = 1$, and set $\phi' \gets \phi_1$.  Else, set $x_1 = 0$ and construct the formula $\phi_0$ by substituting $x_1 = 0$ in $\phi$.  Then set $\phi' \gets \phi_0$.  Now recurse on the formula $\phi'$ to obtain an assignment to the variables in $\phi'$, which are $x_2, \dots x_m$.  This full assignment to $x_1$, and $x_2, \dots, x_m$, is a solution to $\phi$.  Return it.

	Each recursive step calls $M$ once, and solves for one variable.  Thus there are a polynomial number of steps in the input size (linear, in fact).  Since $\P$ is closed under composition, this algorithm takes polynomial time to run.

	% \noindent \textbf{Constructing $\phi_1$.}  Substitution works as follows.  Say we wish to substitute $x_1 = 1$.  Consider any clause $x_1 \wedge x_i \wedge x_j$ where $i, j \neq 1$.  Replace this clause with $\bar{x_i} \wedge x_i \wedge x_j$.  Consider any clause $x_1 \wedge x_i \wedge x_j$.  Replace it with $x_i \wedge x_i \wedge x_j$.  For $x_1 \wedge \bar{x_1} \wedge x_1

    \section*{Problem 3 (3 points)}
	\subsection*{Question}
    Recall $\EXP = \bigcup_{c\ge 1} \TIME(2^{n^c})$ and $\NEXP = \bigcup_{c\ge 1} \NTIME(2^{n^c})$. \\
    Prove that if $\P = \NP$, then $\EXP = \NEXP$.

	\subsection*{Answer}
	Say $\P = \NP$.  Let $L$ be any language in $\NEXP$ on the alphabet $\{0, 1\}$.  Let $L'$ be the language on $\{0, 1, a\}$ (where $a$ is just some other symbol) given by
	$$L' = \{x \cdot a^{2^{|x|}} \mid x \in L\}$$

	That is, $L'$ contains each string $x$ in $L$ followed by a long string of the symbol $a$ repeated $2^{|x|}$ times.
	
	Let $N$ be a nondeterministic Turing machine on the binary alphabet which decides $L$ in $O(2^{n^c})$ time for some $c$.
	We now construct a nondeterministic Turing machine $M$ on the alphabet $\{0, 1, a\}$ to decide $L'$.
	On any input, $M$ runs $N$ on the binary characters in the input and treats input squares with the symbol $a$ as blank.
	Then on input $y = x \cdot a^{2^{|x|}}$, $M$ runs in $O(2^{|x|^c})$.  Thus $M$ runs 
	in $O(|y|^c)$.
	Thus $M$ is a nondeterministic polynomial decider for $L'$.
	By the assumption $\P = \NP$, this means there exists a deterministic polynomial-time Turing machine $R$ which decides $L'$.

	Finally, let $Q$ be a Turing machine on the binary alphabet which, on input $x$, simulates $R$.
	If $R$ tries to move past the last filled bit in $x$, $Q$ simulates that $R$ reads in the character $a$ if the tape head is at position $\leq |x| + 2^{|x|}$, and blank otherwise.
	Then $Q$ runs in $O(\text{the runtime of } R)$, which is polynomial in $|x| + 2^{|x|}$ (the input size to $R$), so exponential in $|x|$.  Thus $Q$ is a deterministic exponential time decider for $L$, so $L \in \EXP$.  Since $L$ was arbitrary, this means $\NEXP = \EXP$.


    \section*{Problem 4 (3 points)}

	\subsection*{Question}
    Define $\mathbf{E}$ to be $\bigcup_{c\ge 1} \TIME(2^{cn})$. Prove that $\mathbf{E} \neq \PSPACE$.\\
    \emph{(Hint: find some property of $\PSPACE$ that doesn't hold of $\mathbf{E}$.)}
	\subsection*{Answer}

	$\PSPACE$ is closed under composition, whereas $\E$ is not.

	To prove that $\E$ is not closed under composition, we perform diagonalization. Consider the language
	\begin{multline*}
		L = \{
		% \langle M \rangle \cdot 1^n \mid M \text{ is a Turing machine which does not accept within } \\ 2^n \text{ steps on all inputs of length } n
		\langle M \rangle \cdot 1^k \mid M \text{ is a Turing machine which does not accept } \\ \text{ on input } \langle M \rangle \cdot 1^k \text{ within } k 2^{c (|M| + k)} \text{ steps for any } c \leq k
	\}
	\end{multline*}
	Say $L \in \E$.  Then there exists a Turing machine $T$ which decides it in $O(2^{nc})$ time.  Say $T$ accepts $\langle T \rangle \cdot 1^k$ for some $k \geq c$ such that $T$ runs in less than $k 2^{nc}$ time on all inputs of length $n > k$.  But then because $T$ runs within $k2^{cn}$ steps, the string $\langle T \rangle \cdot 1^k \notin L$!  Thus it must be that $T$ rejects input $\langle T \rangle \cdot 1^k$ for all such $k$.  But certainly on each of these inputs, $T$ does not accept within $2^{mn}$ steps for any $m \leq k$ (as $T$ does not accept at all).  But this means that $\langle T \rangle \cdot 1^k \in L$.  This is a contradiction.  Thus, it must be that $L \notin \E$.

	The last step to show $\E$ is not closed under composition is to show $L \in \E^\E$. Let
	%  Observe that for each fixed $c$, an $\E$ oracle can tell whether $M$ accepts on $\langle M \rangle \cdot 1^k$ within $k 2^{c(|M| + k)}$ steps, by simply simulating $M$ on this input for $k 2^{c(|M| + k)}$ steps.  Let
	\begin{multline*}
	\bar{L} := \{ \langle M \rangle \cdot 1^k \cdot 0 \cdot 1^n \mid M \text{ is a Turing machine which accepts on input}
	\\ (\langle M \rangle \cdot 1^k) \text{ within } 2^n \text{ steps}
	\}
	\end{multline*}

	Certainly $\bar{L} \in \E$ since we can simply simulate $M$ for $2^n$ steps.  And given an $\bar{L}$ oracle, we can decide $L$ by querying $\bar{L}$ on input $\langle M \rangle \cdot 1^k \cdot 0 \cdot 1^{c (|M| + k)}$ for each $c \leq k$.
	% $= 2^{\log(k) + c(|M| + k)} \leq 2^{2c(|M| + k)} \in O(2^{2c(|M| + k)})$ steps.  (An $\E$ oracle can solve this since it requires simulating fewer than $O(2^{2cn})$ steps in the input size $n$ to the oracle.)  And we only need to call this oracle a linear number of times in the input length to check what happens for each $c \leq k$.  
	Thus $L \in \E^\E$. (In fact, this argument showed that $L \in \P^\E$.  This is the issue with not allowing runtimes with a polyomial in the exponent, like runtime $2^{n^k}$, and only allowing $2^{cn}$: we can write down strings of polynomial size in the input length, but can't perform an exponential operation on them while staying within class $\E$.)
	% [I think the fact that $L \in \P^\E$ indicates something is wrong with this argument, but I don't have time to fix it!  Maybe we need to send in $k$ in binary...]

	Finally, we must show that $\PSPACE$ is closed under composition.  This follows for roughly the same reason $\P$ is closed under composition: if $A$ is a p-space Turing machine for language $X$, and $B$ is a p-space Turing machine for $Y$ which requires an $X$-oracle, we can decide $Y$ in p-space by running $B$, and calling $A$ whenever $B$ calls the $X$-oracle.  (This only requires polynomial space because we can erase the memory used by $A$ each time the oracle is called.  Anything from the computation of $A$ which $B$ needs to retain is something that $B$ would have had to write in its own memory anyway.  Thus, the algorithm only needs at most the amount of memory used by $B$ plus that used by one call to $A$, which is the sum of two polynomials and hence polynomial.)

	% \log(k)}) \in O(2^{(c + 1)(|M| + k)})$

	% This is in $\E^{\E}$, as a $\E$ oracle can check whether $M$ halts on any given input, and we can call this oracle $2^n$ times to check each input.

	% Say $L \in \E$.  Let $T$ be an $O(2^{nc})$ TM which decides $L$.


	% Let $L$ be any language in $\PSPACE$.  Let 
	% $$
	% 	L' := \{x : x^{|x|} \in L\}
	% $$
	% That is, $L$ is the set of strings $x$ where the concatenation $x \cdot x \cdot \dots \cdot x$ of $x$ with itself $|x|$ times is in $L$.

	% First I prove that $L' \in \PSPACE$.  Let $M$ be any Turing machine for $L$. We can construct a Turing machine $M'$ for $L'$ which on input $x$ writes $x^{|x|}$ on the tape and then calls $M$.  $M$ uses polynomial space in $|x^{|x|}| = |x|^2$ which is still polynomial in $|x|$.  (It looks weird that $|x^{|x|}| = |x|^2$, but this is just because we are using multiplication notation rather than addition notation for concatenation.)

	% By the time hierarchy theorem, there exists a $c > 1$ and a language $L$ such that $L \in \TIME(2^{nc}) \setminus \TIME(2^n)$.
	% (This requires that $T(n) = 2^n$ is time constructable, which is clear since you can have a Turing machine write a 1 and then $n$ zeros to express $2^n$ in binary.)
	% Note that $L \in \mathbf{E}$.

	% For contradiction, say $\mathbf{E} = \PSPACE$.  Then, as shown above, $L' \in \PSPACE$, so $L' \in \mathbf{E}$.
	% We know that 
	
	% Since $L \in \mathbf{E}$, let $M$ be a Turing machine for $L'$ which terminates in $O(2^{cn})$ time for every input.



	\section*{Problem 5 (4 points, 2 for each sub-problem)}
	
	\subsection*{Question}

	In this problem we will see an ``amplification'' argument, in which we prove that a weak-looking lower bound actually implies a stronger lower bound. (We love these things.)
	
	\begin{enumerate}[(A)]
		\item Prove that if $\TIME(n^{1+\epsilon})=\TIME(n)$ for some $\epsilon>0$, then $\TIME(n^c) = \TIME(n)$ for all constant $c >1$.\\ 
		(You may assume without proof that the function $f(n) := \lceil n^{1+\epsilon} \rceil$ is time constructible, for every $\epsilon > 0$.)
		
		If you want a hint, see the footnote.\footnote{\tiny First try to show that the assumption implies $\TIME(n^{(1+\epsilon)^2})=\TIME(n^{1+\epsilon}) = \TIME(n)$, then try to iterate the argument.}
		
		\item  Use (A) and the statement of the time hierarchy theorem from class to prove that $\TIME(n)$ is a proper subset of $\TIME(n^{1+\epsilon})$ for \emph{every} $\epsilon > 0$.
	\end{enumerate}
	
	\subsection*{Answer to A}
	Say $\TIME(n^{1+\epsilon}) = \TIME(n)$ for some $\epsilon > 0$.  
	% Observe that if $\epsilon > 1$, this also holds for any $\epsilon \leq 1$.  So WLOG let's take $\epsilon < 1$.

	Let $L \in \TIME(n^{(1 + \epsilon)^2}) = \TIME(n^{1 + 2\epsilon + \epsilon^2})$.

	Let
	$$
	L' = \{x \cdot a^{|x|^{1 + \epsilon}} \mid x \in L\}
	$$
	Here, $L'$ is a language on the alphabet $\{0, 1, a\}$, where $a$ is some other symbol.  ($L$ is a language on alphabt $\{0, 1\}$.)

	For each $x \in L$, $y := x \cdot a^{|x|^{1 + \epsilon}}$ can be decided as being in $L'$ in time $O(|x|^{(1 + \epsilon)^2})$, by simply calling a Turing machine for $L$ on the bits in $y$ before the first $a$.  Equivalently, $L$ can be decided as being in $L'$ in $O(|y|^{1 + \epsilon})$.  Thus $L' \in \TIME(n^{1 + \epsilon})$.  Because $\TIME(n^{1 + \epsilon}) = \TIME(n)$, $L' \in \TIME(n)$.

	Let $M'$ be a Turing machine for $L'$ that runs in $O(n)$.  We can construct a Turing machine $M$ for $L$ which runs in time $O(n^{1 + \epsilon})$ as follows.  Given input $x$, $M$ simulates $M'$ on $x$.  Whenever $M'$ moves the head on the input tape past the last character of $x$, $M$ simulates $M'$ observing symbol $a$ if the head position is less than $|x| + |x|^{1 + \epsilon}$, and simulates it observing a blank square otherwise.  This Turing machine runs in $O(\text{the runtime of } M')$, which is $O((|x| + |x|^{1 + \epsilon})) = O(|x|^{1 + \epsilon})$ (since $M'$ is linear time in its input).  Thus, $M$ is $O(n^{1 + \epsilon})$ so $L \in \TIME(n^{1 + \epsilon})$.  Since $\TIME(n^{1 + \epsilon}) = \TIME(n)$, $L \in \TIME(n)$.

	Thus $\TIME(n^{(1 + \epsilon)^2}) = \TIME(n^{1 + \epsilon}) = \TIME(n)$.  Since $(1 + \epsilon)^2 > 1 + 2 \epsilon$, this means $\TIME(n^{1 + 2 \epsilon}) = \TIME(n)$.  Iterating this argument, now taking $\epsilon$ to be twice our original $\epsilon$, shows that $\TIME(n^{1 + 4 \epsilon}) = \TIME(n)$.  Repeating this lets us show $\TIME(n^{1 + 2^k \epsilon}) = \TIME(n)$ for all $k > 0$.

	For any $c$, there exists a $k$ such that $1 + 2^k \epsilon > c$.  Thus $\TIME(n^c) = \TIME(n)$ for all $c > 1$.

	\subsection*{Answer to B}

	The time hierarchy theorem states that there exists a $c > 0$ such that for any time constructable function $T(n)$, $\TIME(T(n)^c) \not\subseteq \TIME(T(n))$.
	Taking $T(n) = n$, this implies that $\TIME(n^c) \neq \TIME(n)$ for some $c$.
	By part A, this maens that in fact $\TIME(n^{1 + \epsilon}) \neq \TIME(n)$ for all $\epsilon > 0$.  And certainly $\TIME(n) \subseteq \TIME(n^{1 + \epsilon})$, so we must have $\TIME(n) \subsetneq \TIME(n^{1 + \epsilon})$.

	% {\bf Totally optional food for thought:} We know that for Turing machines, 
	% $\TIME(n)$ is a proper subset of $\TIME(n \cdot \log^2 n)$. (This follows from Corollary 9.11 in Prof Sipser's book, for example.) Can you improve this proper containment (reduce the $\log^2 n$ factor) by using an amplification argument?
		
\end{document}