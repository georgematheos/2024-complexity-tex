\documentclass{article}
\usepackage{amsmath}
\usepackage{fullpage}
\usepackage{amsmath,amssymb,verbatim}
\usepackage{algorithm}
\usepackage{algorithmicx}
\usepackage{algpseudocode}
% \usepackage{fullpage}
\usepackage{enumerate}
\usepackage{xcolor}
\renewcommand{\P}{\mathbf{P}}
\newcommand{\NP}{\mathbf{NP}}
\newcommand{\coNP}{\mathbf{coNP}}
\newcommand{\EXP}{\mathbf{EXP}}
\newcommand{\BPP}{\mathbf{BPP}}
\newcommand{\RP}{\mathbf{RP}}
\newcommand{\NEXP}{\mathbf{NEXP}}
\newcommand{\PH}{\mathbf{PH}}
\newcommand{\PSPACE}{\mathbf{PSPACE}}
\newcommand{\TIME}{\mathbf{TIME}}
\newcommand{\DTIME}{\mathbf{DTIME}}
\newcommand{\NTIME}{\mathbf{NTIME}}
\newcommand{\LOG}{\mathbf{LOGSPACE}}
\newcommand{\SIZE}{\mathbf{SIZE}}
\newcommand{\AM}{\mathbf{AM}}
\newcommand{\MA}{\mathbf{MA}}

\newcommand{\mf}[1]{\mathbf{#1}}

\def \F {{\mathbb F}}
\def \N {{\mathbb N}}

\def \ATIME{{\mathsf{ATIME}}}
\def \NTIME{{\mathsf{NTIME}}}
\def \eps {{\varepsilon}}

\def \ASPACE{{\mathsf{ASPACE}}}
\def \SPACE{{\mathsf{SPACE}}}
\def \TIME{{\mathsf{TIME}}}
\def \BPL{{\mathbf{BPL}}}

\def \poly{\text{poly}}
\def \Var{\text{Var}}

\begin{document}
	
	
	\begin{center}
		\Large
		6.541/18.405 Problem Set 3
		
		\vspace{3pt}
		\normalsize
		due on {\bf Thursday, May 2, 11:59pm}
	\end{center}
	
	{\bf Rules:} You may discuss homework problems with other students and you may work in groups, but we require that you {\em try to solve the problems by yourself before discussing them with others}. Think about all the problems on your own and do your best to solve them, before starting a collaboration. If you work in a group, include the names of the other people in the group in your written solution. {\bf Write up your {\em own} solution to every problem}; don't copy answers from another student or any other source. Cite {\bf all} references that you use in a solution (books, papers, people, websites, etc) at the end of each solution. 
	
	We encourage you to use \LaTeX, to compose your solutions. The source of this file is also available on Piazza, to get you started!
	
	{\bf How to submit:} Use Gradescope entry code \textbf{2P3PEN}.\\ \textbf{\large Please use a separate page for each problem.} 

\section*{Problem 1: Circuit Lower Bounds from Pseudorandom Generators (3 Points)}

\subsection*{Question}
We showed that circuit lower bound implied ``good'' PRGs. Here is a sort-of converse:

Show that if there is a $O(\log(n))$-seed 1/10-pseudorandom generator computable in $2^{O(m)}$ time on $m$-bit seeds, then there is an $\eps>0$ such that $\DTIME[2^{O(n)}]\not\subset\SIZE[2^{\eps n}]$.

\subsection*{Answer}

Say there exists a $1/10$ $O(\log(n))$ seed pseudorandom generator $g : \{0, 1\}^* \to \{0, 1\}^*$, computable by a Turing machine $G$ in $2^{km}$ time on $m$-bit seeds.
Say specifically (WLOG) $g$ is defined on all $x$ values of length $c \log(n)$ for $n \in \mathbb{N}$, and $|x| = c \log(n) \implies |g(x)| = n$.

For any string $y$, let $y_{1:l}$ denote the first $l$ bits of $y$.

Consider the decision problem $f : \{0, 1\}^* \to \{0, 1\}$ where
$$
f(x) = 1 \iff \exists y \in \{0, 1\}^{|x| - 1} \text{ s.t. } g(y)_{1:|x|} = x \qquad (*)
$$

For inputs $x$ of size $n$, this can be computed in time $2^{n-1} \cdot 2^{k(n-1)} \in 2^{O(n)}$ by looping over the $2^{n-1}$ possible strings $y$, and computing $g(y)$ in time $2^{k(n-1)}$ for each.

Say for contradiction that $f \in \SIZE[2^n]$.  Let $C$ be the circuit of size $2^n$ that decides this for a given $n$.
Observe that for any $y \in \{0, 1\}^{n-1}$, $C(g(y)_{1:n}) = 1$.
This is because $g(y)_{1:n}$ is in the range of $g$ on $\{0, 1\}^{n-1}$, which is the very property thatn decision problem $f$ checks, and $C$ is a circuit that computes $f$.
Thus
$$
\Pr_{y \in \{0, 1\}^{n-1}}[C(g(y))_{1:n} = 1] = 1 \quad (*)
$$
However,
$$
\Pr_{x \in \{0, 1\}^{2^{n-1}}}[C(x_{1:n}) = 1] \leq \frac{1}{2} \quad (**)
$$
becuase here, the strings $x_{1:n}$ cover all possible $n$-long bit-strings,
and only half of these can possibly be in the image of $g$ on inputs in $\{0, 1\}^{n-1}$.

Statements (*) and (**) contradict that $g$ is a 1/10 PRG,
since they mean a circuit $C$ of size $2^n$ distinguishes between the distribution
induced by $g$, and the actual uniform distribution on strings of size $2^{n-1}$.
% Q: am I off by a factor of 2 here???

% \bigskip
% Say there exists such a pseudorandom generator $g : \{0, 1\}* \to \{0, 1\}*$ computed
% by Turing machine $G$ running in $2^{km + l}$ time.
% WLOG, say that for all $n$, $|x| = c \log(n) \implies |g(x)| = n$.
% For any circuit $C$ of size $n$,
% $$
% |Pr_{x \in \{0, 1\}^{c \log(n)}}[C(g(x)) = 1] - Pr_{x \in \{0, 1\}^n}[C(x) = 1] | < 1/10
% $$

% The idea is going to be that in time $2^m \times 2^{km} = 2^{(k+1)m}$, a Turing machine can run $g$ on every length $m$ string, and compute $Pr_{x \in \{0, 1\}^{c \log(n)}}[C(g(x)) = 1]$ exactly, but no circuit of size $2^m$ can do this.

% % Fix a sufficiently large $n$, and let $x^* \in \{0, 1\}^{c \log(n) + 1}$ such that $$x^* \notin \{g(y) : y \in \{0, 1\}^{c \log(n)}\}$$

% Consider the decision problem $f : \{0, 1\}^* \to \{0, 1\}$ where
% $$
% f(x) = 1 \iff \exists y \in \{0, 1\}^{|x| - 1} \text{ s.t. } g(y) = x \qquad (*)
% $$
% Suppose for contradiction that $f \in \SIZE[2^n]$.
% Let $C$ be the circuit of size $2^n$ that decides this for a given $n$.
% Say $m = cn$ (so $m = c\log(2^n)$).
% Observe
% $$
% Pr_{y \in \{0, 1\}^m}[C(g(y)_{1:m+1}) = 1] = 1
% $$
% Here, $g(y)_{1:m+1}$ is the first $m+1$ bits of $g(y)$.

% However, 
% $$
% Pr_{x \in \{0, 1\}^{2^n}}[C(x_{1:m+1}) = 1] \leq 1/2
% $$
% because only half of the $m+1$ bit strings can be in the range of a function with domain $\{0, 1\}^m$.
% This contradicts $(*)$.
% Thus $f \notin \SIZE[2^n]$.
% But certainly $f \in \DTIME[2^{O(n)}]$, because it is possible to loop over all $2^n$ $y$ values of length $n$, and for each one check if $x = g(y)$ in time $2^{O(n)}$.  This yields an algorithm for deciding $f$ with total runtime $2^n 2^{O(n)} = 2^{O(n) + n} = 2^{O(n)}$.

% Here, $C$ is the description of a circuit of size $2^m$, and $m$ and $s$ are integers.


\newpage
\section*{Problem 2: Randomized Approximate Counting with an NP Oracle (12 pts, 3 for each sub-problem)}

\subsection*{Question}

We will develop a real-life application of SAT solvers.
{\bf Assume $\mathbf{P}=\mathbf{NP}$ in this question.} Let $H_{n,k}$ be a pairwise independent hash family of functions from $\{0,1\}^n$ to $\{0,1\}^k$.
\begin{itemize}
	\item[(a)] Prove that there is a constant $p \in (0,1)$ and a constant $\eps > 0$ such that for every $k$ and  $S \subseteq \{0,1\}^n$, 
	
	\begin{itemize} 
	\item[$\bullet$] if $|S| \leq 2^{k-2}$, then \[\Pr_{h \in H_{n,k}}[\text{there is an $x \in S$ such that $h(x) = 0^k$}] < p - \eps,\] and
	\item[$\bullet$] if $|S| \geq 2^{k-1}$, then \[\Pr_{h \in H_{n,k}}[\text{there is an $x \in S$ such that $h(x) = 0^k$}] > p + \eps.\] 
\end{itemize}
		\item[(b)]
		Use part (a) to show that that there is a randomized polynomial-time algorithm that approximates \#\textsf{SAT} within a factor of 4. More precisely, there is a randomized polynomial-time algorithm that given any Boolean formula $F$ outputs a number $K$ such that $\#\textsf{SAT}(F)/2 \leq K \leq 2\cdot \left(\#\textsf{SAT}(F)\right)$.
	\item[(c)]
	Show that for any constant $\varepsilon>0$, there is a randomized polynomial-time algorithm that approximates $\#\textsf{SAT}$ within a factor of $1+\varepsilon$. (Hint: Try to modify the given formula $F$ in some natural way that changes the number of SAT assignments, then feed the modification to your algorithm from part (b).)
	\item[(d)]
		Show that you can derandomize the algorithm.	That is, prove that if $\P = \NP$ then for every function $f\in \#\mathbf{P}$, and any constant $\varepsilon>0$, there is a deterministic polynomial-time algorithm that approximates $f$ within a factor of $1+\varepsilon$. (Warning: the approximate counting problem is \emph{not} a decision problem, so you cannot just ``plug in'' $\P = \NP$ here\dots)
\end{itemize}

\newpage
\subsection*{Answer}
\subsubsection*{Answer to (a)}

Say $|S| \leq 2^{k - 2}$.
For $x \in S$, let $A_x$ denote the event $A_x = \{h : h(x) = 0^k\}$.
Then 
$$
	\Pr[\exists x \in S \text{ s.t. } h(x) = 0^k]
	= \Pr[\cup_{x \in S} A_x]
	\leq
	\sum_{i=1}^{|S|} \Pr[A_{x_i}]
	= \sum_{i=1}^{|S|} \frac{1}{2^k}
	= |S|/2^k
	\leq 1/4
$$

\medskip
\noindent
Now say $S'$ is a set such that $|S'| \geq 2^{k-1}$.
Let $S \subseteq S'$ be a subset of size exactly $2^{k-1}$.
I will show that $\Pr[\cup_{x \in S}A_x] \geq k$ for a certain $k$.
Since $S \subseteq S'$, this will imply $\Pr[\cup_{x \in S'}A_x] \geq k$.

Let $T := \sum_{x \in S}1_{A_x}$, the number of $x$ values in $S$ such that $h(x) = 0^k$.
Then $E[T] = |S|/2^k = 1/2$, and
since the indicators $1_{A_x}$ are pairwise independent random variables,
$$
\text{Var}[T] = \sum_{x \in S} \text{Var}[1_{A_x}]
= \sum_{x \in S} 2^{-k}(1 - 2^{-k})
= 2^{-1}(1 - 2^{-k}) = \frac{1}{2} - \frac{1}{2^{k+1}}
$$
I wish to upper bound $\Pr(\cup_{x \in S} A_x) = \Pr(T \geq 1)$.
I proceed as follows, using Cantelli's inequality:
\begin{multline*}
\Pr(T = 0) =
\Pr(T \leq 0) = \Pr(T - \frac{1}{2} \leq -\frac{1}{2}) = \Pr(T - E[T] \leq -\frac{1}{2})
\leq \frac{\text{Var}[T]}{\text{Var}[T] + (\frac{1}{2})^2} \\
= \frac{\frac{1}{2} - \frac{1}{2^{k+1}}}{
	\frac{1}{2} - \frac{1}{2^{k+1}} + 1/4
} \leq \frac{1/2}{1/2 + 1/4} = 2/3
\end{multline*}
Thus,
$$
\Pr(\exists x \in S \text{ s.t. } h(x) = 0^k) = \Pr(T \geq 1) = 1 - \Pr(T = 0) \geq 1/3
$$

\medskip
\noindent
Take $p = \frac{7}{24}$ and $\epsilon = \frac{0.99999}{24}$.
Then when $|S| \leq 2^{k-2}$, $\Pr[\exists x \in S . h(x) =0^k] \leq 1/3 < p - \epsilon$
and when $|S| \geq 2^{k-1}$, $\Pr[\exists x \in S . h(x) =0^k] \geq 1/4 > p + \epsilon$.

\subsection*{Answer to (b)}

Let $m$ be a constant whose value we will set below.

\begin{algorithm}{$\mathsf{RandomizedApproximateCounting}$}
\begin{algorithmic}[1]
\Require Formula $F$ on $n$ variables.
\For{$k = 0, \dots, n$} \label{line:inner_for}
	\State Initialize a count $c$ to $0$.
	\State $N \gets \left \lceil{(n + m)/\epsilon^2}\right \rceil$
	\For{$i = 0, \dots, N$}
		\State Generate a pairwise independent hash function $h : \{0, 1\}^n \to \{0, 1\}^k$.
		\State Construct a formula $\phi \gets (x \mapsto [F(x) \wedge h(x) = 0^k])$.
		\State Use a polynomial time SAT solver to compute $s_\phi \gets \mathsf{SAT}(\phi)$.
		\State If the $s_\phi = 1$, increment $c$.
	\EndFor \label{line:end_inner_for}
	\State If $c / N < p$, return $2^{k-2}$. \label{line:return}
\EndFor
\State Return $n$.
\end{algorithmic}
\end{algorithm}

The algorithm $\mathsf{RandomizedApproximateCounting}$ returns a value $2^k$ which,
with probability $> 2/3$, approximates $\#\mathsf{SAT}(F)$ within a factor of 4.

The proof of this has two parts.
First, we must show that if $\#\mathsf{SAT}(F) < 2^{k-1}$,
it is very unlikely that the algorithm returns the value $2^k$.
Second, we must show that if $\#\mathsf{SAT}(F) > 2^{k+1}$,
it is very likely that the algorithm returns $2^k$
or a smaller value (in which case it is guaranteed to reach the part of
the for loop where it will return a value $2^{k+1}$ or greater).

Say $\#\mathsf{SAT}(F) \geq 2^{k-1}$  and consider the $k$th iteration of the
outer for loop.
By part (a), taking $S$ to be the set of assignments satisfying $F$,
the probability that $s_\phi = 1$ is $\geq p + \epsilon$
at any iteration of the inner loop.
Let $c_k$ denote the value of $c$ in the algorithm on iteration $k$.
Then $\Pr[c_k < Np] \leq \Pr_{v \sim \text{Binomial}(p + \epsilon, N)}[v < Np]$.
By Hoeffding's inequality, this implies
$$
\Pr[c_k < Np] \leq e^{-2N\epsilon^2}
$$
Since $N \geq \frac{n + m}{\epsilon^2}$,
$$
\Pr[c_k < Np] \leq e^{-2(n + m)}
$$
Thus, the probability that the algorithm returns in an iteration $k$
where $2^{k-1} \leq \#\mathsf{SAT}(F)$ is
$$
\Pr[\exists k \text{ s.t. }2^{k-1} \leq \#\mathsf{SAT}(F) \text{ and } c_k < Np] = 1 - \prod_{k=0}^{\log(\#\mathsf{SAT}(F)) + 1}{\Pr[c_k \geq Np]}
$$
where this equality follows because the result of the algorithm at different iterations of the outer loop are all statistically independent.
We can continue bounding this:
\begin{multline*}
1 - \prod_{k=0}^{\log(\#\mathsf{SAT}(F)) + 1}{\Pr[c_k \geq Np]}
\leq 1 - \prod_{k = 0}^n{\Pr[c_k \geq Np]} \\
= 1 - \prod_{k = 0}^n{(1 - \Pr[c_k < Np])} 
\leq 1 - (1 - e^{-2(n + m)})^n
\leq ne^{-2(n + m)}
< 2/3
\end{multline*}
This inequality holds for all $n$ if we choose a sufficiently large $m$.
This guarantees that it is unlikely we return a value $2^{k-2} \leq \frac{1}{2}\#\mathsf{SAT}(F)$.

\medskip
Now we need to prove that 
if $\#\mathsf{SAT}(F) \leq 2^{k-2}$ the algorithm has a high probability of returning
$2^{k-2}$.
This means that on the first iteration of the loop with a value $k$ such that
$2^{k-2} \geq \#\mathsf{SAT}(F)$, ie. where $2^k$ is significantly larger
than $\#\mathsf{SAT}(F)$, it is almost certain that
the loop will return the value $2^{k-2}$.
This guarantees that it is very unlikely
the algorithm returns a value $2^{k-1} \geq 2\#\mathsf{SAT}(F)$
(which is what would occur at the next iteration of the outer loop,
if the algorithm failed to detect when $\#\mathsf{SAT}(F) \leq 2^{k-2}$).
Combining this with the above result that it is very unlikely
the algorithm, returns $2^{k-2} \leq \frac{1}{2}\#\mathsf{SAT}(F)$,
we derive that with high probability, the algorithm returns
a value $2^{k-2} \in (\frac{1}{2}\#\mathsf{SAT}(F), 2\#\mathsf{SAT}(F))$.

For this part of the proof, note that by part (a),
$\#\mathsf{SAT}(F) \leq 2^{k-2}$ implies that
$\Pr[s_\phi = 1] > p - \epsilon$ in the algorithm,
so by the same analysis as last time using the Hoeffding bound,
$\Pr[c_k \geq Np] < e^{-2(n + m)} < 1/3$.
(As before, getting this lower bound depends upon choosing a large enough constant $,m$.)
% NOTE to self - presumably we can drive this error
% down to be arbitrarily low, by playing with the constants here.
Thus it is highly likely that $c_k < Np$ so the algorithm will return (line 10)
in any iteration of the loop where $\#\mathsf{SAT}(F) \leq 2^{k-2}$,
if it reaches such an iteration.

\subsection*{Answer to (c)}

Let $F$ be a formula on variables $x_1, \dots, x_n$.
For $i = 2, \dots, n$, let $F_i$ be a formula, identical to $F$, but on a distinct set of variables $y^i_1, \dots, y^i_n$.

Let $G_i = F \wedge F_2 \wedge \dots \wedge F_i$, a formula on $ni$ variables.
Observe that if $F$ has $s$ satisfying assignments, $G_i$ has $s^i$ satisfying assignments.

Choose $i$ large enough that $(1 + \epsilon)^i > 4$.
Let $s_i$ denote the number of satisfying assignments to $G_i$.
Run our algorithm from part (b) on $G_i$ to find a value $K$ such that
$K \in [s_i/2, 2s_i]$.
Then $s_i \in [K/2, 2K]$.
Thus $s^i \in [K/2, 2K]$.
Let $k = K/2$, so $s^i \in [k, 4k]$.
Then $$s \in [(k)^{1/i}, 4^{(1/i)}(k^{1/i})] \subseteq [(k)^{1/i}, (1 + \epsilon)(k^{1/i})]$$
Thus, this has yielded a $(1 + \epsilon)$ approximation to $\#\mathsf{SAT}(F)$.

\subsection*{Answer to (d)}
First, observe that if $\P = \NP$, the polynomial hierarchy collapses, so since $\BPP \subseteq \PH$,
we have $\BPP \subseteq \P$.
I do not know a way to directly use this fact to answer this question, but I will give a derandomization based on a similar idea to how we proved $\BPP \subseteq \PH$.

Consider lines \ref{line:inner_for}-\ref{line:end_inner_for} in $\mathsf{RandomizedApproximateCounting}$.
Let $A(F, k, r)$ denote the Turing machine that runs these lines of the algorithm for loop iteration $f$, given formula $F$, invoking randomness $r$,
and outputting the indicator $1_{c/N < p}$. 
(That is, $A$ outputs 1 iff $c/N < p$ when line $\ref{line:end_inner_for}$ of the algorithm is reached.)

In part (b), we proved that If $|S| \leq 2^{k-2}$ then with probability $\geq 2/3$, $A(F, k, r) = 1$, and if $|S| \geq 2^{k-1}$ then with probability $\geq 1/3$, $A(F, k, r) = 0$.
By repetition, with $O(n)$ iterates, we can boost these probabilities to $1 - 2^{-n}$ and $2^{-n}$.
Let $A'$ be a (still polynomial probabilistic time algorithm) which is boosted to probabilities $1 - 2^{-100n}$, $2^{-100n}$ for inputs of length $n$.

Let $M_n$ be the set of $n$-strings $(F, k)$ such that the resulting set $S$ satisfies $|S| \geq 2^{k-1}$.
For any $x \in M_n$, $\Pr_r[A'(x, r) = 1] < 2^{-100n}$.
Since $|M_n| \leq 2^n$,
Thus $\Pr_r[\exists x \in M_n . A'(x, r) = 0] < 2^n 2^{-100n} = 2^{-99n}$.
Let $M'_n$ be the set of $n$-strings $(F, k)$ such that the resulting set $S$ satisfies $|S| \leq 2^{k-2}$.
By the same logic, $\Pr_r[\exists x \in M'_n . A'(x, r) = 0] < 2^{-99n}$.
Thus
$$
\Pr_r[\exists x \in M'_n . A'(x, r) = 0 \vee \exists x \in M_n . A(x, r) = 1] < 2^{-99n + 1} < 1
$$
Thus there is some random string $r^*$ such that
$$
\forall  x \in M'_n . A'(x, r^*) = 1 \wedge \forall x \in M_n . A'(x, r^*) = 0 \quad (*)
$$

If $\P = \NP$ (and hence $\P = \PH$), I claim it is possible to in fact construct such a random string $r^*$ in polynomial time.
For bitstring $y$, let $T_y$ be the statement ``There exists a bitstring $z$ such that (*) holds with $r^* := y \cdot z$.''
Since we can upper bound the length of $r^*$, the quantification for $z$ in this statement can be quantification of a fixed length.
Thus the statement $T_y \in \PH$ (it requires only 2 levels of quantification),
and so can be decided in polynomial time.
Thus, we can apply the idea underlying the $\mathsf{SearchSAT}$ to $\mathsf{SAT}$ reduction to find a string $r^*$ with property (*) in polynomial time.
(We first check if such an $r^*$ exists where the first bit is 1, then set the first bit accordingly.  Then we find the second bit, and so on.)

Let $B$ be the deterministic polynomial time algorithm which on input $(F, k)$ first finds a string $r^*$ such that satisfies property (*), and then runs $A'(F, k, r^*)$ and returns the result.
Algorithm $B$ is a deterministic algorithm which yields 1 on any $(F, k)$ such that $|S| \leq 2^{k-2}$ and which yields 0 on any $(F, k)$ such that $|S| \geq 2^{k-1}$.

Thus, if we replace lines \ref{line:inner_for}-\ref{line:end_inner_for} in $\mathsf{RandomizedApproximateCounting}$ with algorithm $B$, this yields a derandomized algorithm which is also guaranteed to return a multiplicative 4-approximation to $\#\mathsf{SAT}$.
The same boosting argument from part (c) can then be applied to boost this to an arbitrary $(1 + \epsilon)$ approximation.

% Let $f$ be the function so $f(k, F) = 1$ iff, when $k = 1$, after lines \ref{line:inner_for}-\ref{line:end_inner_for} in $\mathsf{RandomizedApproximateCounting}(F)$, the count $c$ satisfies $c/N < p$ with probability at least $2/3$.
% Observe that $f \in \BPP$, since $f$ is 

% lines~-\ref{line:return} in 
% implement 

\newpage
\section*{Problem 3: Constant Round Arthur-Merlin Collapses (3 points)}

\subsection*{Question}
Prove that for every fixed positive integer $k$, $\AM[k]\subseteq \AM[2]$.

\medskip

\emph{Hint: Try error-reduction, to make the probability of error very small.}


\subsection*{Answer}

WLOG assume $k$ is even. Consider an AM[k] protocol to decide $f(x)$.  Say on any iteration, a sequence of messages $r_1, m_1, r_2, m_2, \dots, r_{k/2}, m_{k/2}$ are sent, where $r_i$ are the random messages from the verifier, and $m_i$ are the messages from the prover.
% Let $M$ be an upper bound (possibly dependent on the size of the input $x$ to the Arthur Merlin protocol) on the total size of the prover's mesasges,
% $M \geq \sum_{j} |m_{j}|$.
Fix a given prover $P$.  Let $A_\text{acc}$ denote the event that the $r_i$ result in $V$ accepting, and let $A_\text{rej}$ denote the event that the $r_i$ result in $V$ rejecting.
By the definition of an AM protocol, if $f(x) = 1$ there is a prover so $\Pr[A_\text{acc}] > 2/3$ and if $f(x) = 0$ then for every prover, $\Pr[A_\text{rej} < 1/3]$.

By the error reduction lemma, by running this protocol in parallel $O(k)$ times (with a minor modification to be described in a moment), we obtain an AM[k] protocol such that if $f(x) = 1$, there is a prover so $\Pr[A_\text{acc}] > 1 - 2^{-k}$, and if $f(x) = 0$, for every prover, $\Pr[A_\text{rej}] < 2^{-k}$.

% The following is the modification to simply running the original protocol in parallel that is needed for this to work (or, at least, I have not seen how error reduction could work without this).  We also now need the verifier to check that it is possible that the messages the prover is sending are possible from a single parallel running many times in parallel (so that the new ``parallel track prover'' isn't cheating by respoding to each )
% same prover is being run in parallel.  That is, if two parallel tracks send the same initial sequence of random bits, the prover's messages up to that point must be identical in both branches, for the verifier to accept.)
\textbf{[TODO: do we need to prove this lemma?]}

Now, consider the following $\AM[2]$ protocol.  On the first round,
the verifier sends a sequence of random bits $r_1, r_2, \dots, r_{k/2}$,
where each sequence of bits is as long as the longest sequence that $r_i$ could have been in the $\AM[k]$ protocol with exponential error bounds.
The prover will send a message which is a concatenation $m_1, m_2, \dots, m_{k/2}$, and the verifier will accept if $r_1, m_1, r_2, m_2, \dots, m_{k/2}$ would have been an acceptable transcript in the $\AM[k]$ protocol with exponential error bounds.

\medskip
\noindent \textbf{Analysis.} If $f(x) = 1$, then there is a prover which would have almost certainly been accepted in the $\AM[k]$ version of the protocol, and running it in this $\AM[2]$ will succeed with equally high probability.
Now say $f(x) = 0$.  Let $r$ denote the full string of randomness sent by the verifier and let $m$ denote the prover's full response.
It is sufficient to upper-bound $\Pr_r[\exists m . V(x, m, r) = 1]$.
By the union bound,
$$
\Pr_r[\exists m . V(x, m, r) = 1] \leq \sum_{m}{\Pr_r[V(x, m, r) = 1]} \leq \sum_m{2^{-k}} \leq 2^M/2^k
$$
where $M$ is an upper bound on the length of $m$, and $k$ is the value we chose in the error bound for the reduced-error $\AM[k]$ protocol.
If we choose to set $k > M + 2$, we get 
$$
\Pr_r[\exists m . V(x, m, r) = 1] < 1/4
$$
which is certainly sufficient.


\newpage

\section*{Problem 4: AM Protocol for Set Lower Bound (6 Points, 2 for each sub-problem)}

\subsection*{Question}

In this problem, we will develop an $\AM$ protocol for proving a set lower bound, which is used as a subroutine in the $\AM$ protocol for graph non-isomorphism. In a set lower bound protocol, the prover needs to prove to the verifier that given a (large) set $S\subseteq \{0,1\}^m$ (where membership in $S$ is efficiently verifiable), $S$ has cardinality at least $K$, up to a factor of $2$. More precisely, given any $K$,
\begin{itemize}
\item  if $|S| \ge K$ then the prover can make the verifier accept with high probability; 
\item if $|S| < K/2$ then the verifier rejects with high probability regardless of what the prover does.
\end{itemize}


\begin{enumerate}
    \item[(a)] Let $H_{m,k}$ be a pairwise independent hash family of functions from $\{0,1\}^m$ to $\{0,1\}^k$. Use the pairwise independent hash family $H_{m,k}$ to give a $2$-round $\AM$ protocol for the set lower bound problem described above.
    
    \item[(b)] Show that there exists an $\AM$ protocol for set lower bound with perfect completeness.

    \emph{Hint: Consider the case where the prover uses multiple hash functions $h_1,\dots, h_n$ so that $\bigcup_{i = 1}^n h_i(S) = \{0,1\}^k$.}
    
    \item[(c)] Generalize the idea from part (b) to show that every problem in $\MA$ has a protocol with perfect completeness. Namely, show that for every language $L\in \MA$, there exists a probabilistic polynomial time verifier $V$ such that 
    \begin{enumerate}
        \item[-] If $x\in L$, then there exists $m$ such that $\Pr_r[V(x,r,m) = 1] = 1$.
        \item[-] If $x\not\in L$, then for all $m$, $\Pr_r[V(x,r,m)]\le 1/3$.
    \end{enumerate}
\end{enumerate}

\newpage
\subsection*{Answer to (a)}
\textbf{The protocol.}
Choose $k$ so that $K \in [2^{k-2}, 2^{k-1}]$.
Let $\mathcal{H}$ be a family of pairwise independent hash functions from $\{0, 1\}^m$ to $\{0, 1\}^k$.
Let $N = 600$.
The verifier sends $N$ random hash function $h_1, \dots, h_M$ to the prover,
and $N$ independent values chosen uniformly at random from $\{0, 1\}^k$, $y_1, \dots, y_N$.
For each $i = 1, \dots, N$, the prover will either say that there is no $x \in S$ so that $h_i(x) = y_i$, or it will send a value $x_i \in S$ such that $h_i(x_i) = y_i$.
The verifier will accept iff $h_i(x_i) = y_i$ for all $i$ where an $x_i$ was sent,
and the number $m$ of values $x_i$ that the prover sent satisfies
$$
m > \frac{7}{6} \frac{K}{2^{k+1}} N
$$
Below we will define $p := \frac{7}{6} \frac{K}{2^{k+1}}$, so we can write this condition as $m > Np$.

\medskip
\noindent
\textbf{Analysis.}
Let $A_y := \{h : \exists x \in S . h(x) = y\}$.
Recall $K \in [2^{k-2}, 2^{k-1}]$.
If $|S| \leq K/2$,
$$
\Pr[A_y] \leq \sum_{x \in S}\Pr_h[h(x) = y] = \frac{|S|}{2^k} \leq \frac{K}{2^{k+1}}
$$

To analyze the case $|S| \geq K$, I will use Cantelli's inequality again.
To lower bound $\Pr[A_y]$ in this case, it suffices to lower bound $\Pr[A_y]$
assuming $|S| = K$, since $\Pr[A_y]$ is nondecreasing in $|S|$.
Henceforth assume $|S| = K$.
Let $A_{x, y} := \{h : h(x) = y\}$.  Observe $A_y = \cup_x A_{x, y}$.
Let $T = \sum_{x \in S} 1_{A_{x, y}}$.
Observe $E[T] = K/2^k$.
Further,
$$
\Var[T] = \sum_{x \in S} \Var[1_{A_{x, y}}] = K \Var[1_{A_{x, y}}] = K(2^{-k}(1 - 2^{-k})) = \frac{K}{2^k}(1 - 2^{-k})
$$
The first equality follows from the decomposition of variance for pairwise independent random variables (here, the $1_{A_{x, y}}$), and the third equality follows from the fact that $1_{A_{x, y}}$ is a $\text{Bernoulli}(2^{-k})$ random variable.

We now apply Cantelli's inequality to lower bound $\Pr[A_y] = \Pr[T \geq 1] = 1 - \Pr[T = 0]$.
By Cantelli's inequality,
\begin{multline*}
\Pr[T = 0] = \Pr[T \leq 0] = \Pr[T - \frac{K}{2^k} \leq -\frac{K}{2^{k}}]
= \Pr[T - E[T] \leq -K/2^k] \\
\leq \frac{\Var[T]}{\Var[T] + (\frac{K}{2^k})^2}
 = \frac{
	\frac{K}{2^k}(1 - 2^{-k})
 }{
	\frac{K}{2^k}(1 - 2^{-k}) + \frac{K}{2^k} \cdot \frac{K}{2^k} 
 } =
 \frac{1 - 2^{-k}}{1 - 2^{-k} + K2^{-k}}
 = \frac{2^k - 1}{2^k - 1 + K}
\end{multline*}
Thus
$$
\Pr[A_y] - 1 - \Pr[T = 0] \geq 1 - \frac{2^k - 1}{2^k - 1 + K}
= \frac{K}{2^k + K - 1}
$$
Now, plugging in $K \leq 2^{k-1}$,
$$
\Pr[A_y] \geq \frac{K}{2^k + K - 1} \geq \frac{K}{2^k + 2^{k-1} - 1}
= \frac{K}{3 \cdot 2^{k-1} - 1}
= \frac{4}{3}\frac{K}{2^{k+1}} \times \frac{3 \cdot 2^{k-1}}{3 \cdot 2^{k-1} - 1}
\geq \frac{4}{3}\frac{K}{2^{k+1}}
$$

Thus if $p = \frac{7}{6}\frac{K}{2^{k+1}}$ and $\epsilon = \frac{0.9999}{6}\frac{K}{2^{k+1}}$,
$$
|S| \leq K/2 \implies \Pr[A_y] < p - \epsilon
$$
and
$$
|S| \geq K \implies \Pr[A_y] > p + \epsilon
$$

Thus, by the Hoeffding bound, if $|S| \geq K$, $\Pr[m > p N] \leq e^{-2N \epsilon^2}$.
So if we take $N \geq \frac{1}{\epsilon^2}$, $\Pr[m > p N] \leq e^{-2} < 1/3$.
Likewise, if $|S| \leq K/2$, $\Pr[m \leq pN] < 1/3$.

Finally, notice that since $\frac{0.9999}{6}\frac{K}{2^{k+1}} \leq \frac{0.9999}{6} \frac{2^{k-1}}{2^{k+1}} = \frac{0.999}{24}$,
it suffices to take $N = 600$.

\subsection*{Answer to (b)}
\textbf{The protocol.} 
Let $n$ be an integer such that
$$
\frac{4 (n \log(K) + 2)}{2^{n+1}} < \frac{1}{100}
$$
Let $k$ be such that $K^n \in [2^{k-2}, 2^{k-1}]$.
Let $S^n$ denote the set $S^n = \{x_1 \cdot x_2 \cdot \dots \cdot x_n : \forall i . x_i \in S\}$.
In the protocol,
the prover will first send $l = 4k$ hash functions,
$h_1, \dots, h_l$, each s.t. $h_i : \{0, 1\}^{nm} \to \{0, 1\}^k$,
to the verifier.
The prover is claiming that for any $y \in \{0, 1\}^k$,
there exists an $i$ and a $x \in S^n$ s.t. $h_i(x) = y$.
The verifier then sends one uniformly random string $y \in \{0, 1\}^k$ to the prover.
The prover sends back a pair $(i, x)$ with $i \in \{1, \dots, l\}$ and $x \in S^n$.
If $h_i(x) = y$, the verifier accepts; otherwise it rejects.

\medskip
\noindent
\textbf{Analysis.}

First, say $|S| \leq K/2$.
Then $|S^n| \leq K^n/2^n$.
Thus for any $i$, $|h_i(S^n)| \leq K^n/2^n$.
Thus
$$
|\cup_{i=1}^l h_i(S^n)| \leq \sum_{i=1}^l |h_i(S^n)|
\leq \sum_{i=1}^l K^n/2^n =
l K^n/2^n
$$
Thus
$$
\frac{1}{2^k}|\cup_{i=1}^l h_i(S^n)| \leq \frac{lK^n}{2^{n + k}}
= \frac{4k}{2^{n}} \frac{K^n}{2^k}
\leq \frac{4k}{2^{n}} \frac{2^{k-1}}{2^k}
= \frac{4k}{2^{n+1}}
$$
Observe that since $K^n \geq 2^{k-2}$, $k-2 \leq \log(K^n)$
so $k \leq \log(K^n) + 2$.

Thus
$$
\frac{1}{2^k}|\cup_{i=1}^l h_i(S^n)| \leq 4(n\log(K) + 2)/2^{n+1} < 1/100
$$
where the last inequality follows due to our choice of sufficiently large $n$.

This shows that the fraction of the values in $\{0, 1\}^k$ covered by the sets $h_i(S)$ is les than $1/100$.
Thus, regardless of what $l$ hash functions the prover sends to the verifier,
if $|S| \leq K/2$, with probability $99/100$, the prover will not be able to produce
an $(i, x)$ pair such that $h_i(x) = y$ for the random $y \in \{0, 1\}^k$ the verifier sends to the prover.

\medskip
Now, say $|S| \geq K$.
To show that the protocol has perfect completeness, our goal is to show that
in this case, there exist some hash functions $h_1, \dots, h_n$ such that
$\forall y \in \{0, 1\}^k, \exists i \in \mathbb{Z}_l, x \in S^n . h_i(x) = y$.
Then, the prover can simply send these hash functions,
and is guaranteed to be able to respond successfully to any query
the verifier makes (that is, any $y$ it sends.)
To prove this I will use the probabilistic method, and show that if 
the hash functions are generated uniformly at random, there is nonzero probability
of obtaining a set $h_{1:l} := \{h_1, \dots, h_l\}$ satisfying this condition.

Let $A_{i, y} := \{h_{1:l} : \exists x . h_i(x) = y\}$.
The goal is to show that
$$
\Pr_{h_{1:l}}[\cap_y \cup_i A_{i, y}] > 0
$$
First, letting $A_{i, y}^C$ denote the complement of $A_{i, y}$, note that
$$
\Pr[\cap_y \cup_i A_{i, y}] = 1 - \Pr[\cup_y \cap_i A_{i, y}^C]
\geq 1 - \sum_{y \in \{0, 1\}^k} \Pr[\cap_i A_{i, y}^C] = 1 - 2^k\Pr[\cap_i A_{i, 0^k}^C] \quad (*)
$$
Because the $h_i$ are independent, 
$
\Pr[\cap_i A_{i, 0^k}^C]
= \Pr[A_{1, 0^k}^C]^l
$.
Our analysis from part (a) shows that
$
\Pr[A_{1, 0^k}] \geq \frac{4}{3} \frac{K}{2^{k+1}}
$
and because $K \geq 2^{k-2}$, this implies
$
\Pr[A_{1, 0^k}] \geq \frac{4}{3} \frac{1}{8} = \frac{1}{6}
$.
Thus $\Pr[A_{1, 0^k}^C] \leq \frac{5}{6}$, so 
$
\Pr[\cap_i A_{i, 0^k}^C] \leq (\frac{5}{6})^l
$.
Combining this with (*), we see
$$
\Pr[\cap_y \cup_i A_{i, y}] \geq 1 - 2^k \Pr[\cap_i A_{i, 0^k}^C] \geq 1 - 2^k (\frac{5}{6})^l
= 1 - 2^k (5/6)^{4k} > 1 - 2^k (1/2)^k = 0
$$
The second inequality here plugged in $l = 4k$, which we set at the beginning of the protocol.
We chose $4k$ specifically because $(5/6)^4 < 1/2$.

This strict inequality $\Pr[\cap_y \cup_i A_{i, y}] > 0$ completes the proof.

\subsection*{Answer to (c)}
Consider a decision problem $f : \{0, 1\}^* \to \{0, 1\}$ decided by a Merlin-Arthur protocol with verifier $V$.
WLOG, say that the transcripts for this protocol use exactly $kn^k$ random bits on inputs of size $n$.
Fix an input $x$ and an optimal prover $P$.
Let $n = |x|$ and let $R = kn^k$.
Let $S \subseteq \{0, 1\}^{R}$ denote the set of random strings on which $(V, P)$ will accept $x$.
It is guaranteed that either $|S| \geq \frac{2}{3}2^R$ and $f(x) = 1$, or $|S| \leq \frac{1}{2}2^R$ and $f(x) = 0$ (this is the ``BPP promise'' for the interactive proof protocol).
We can slightly adapt the idea from part (b) to have the prover prove to the verifier, with perfect completeness, that $|S| \geq \frac{2}{3}2^R$.

\medskip
\noindent
\textbf{The protocol.}
Identically to as in the protocol in part (b), the prover first sends $l$ hash functions $h_1, \dots, h_l$.
The verifier then sends a random value $y \in \{0, 1\}^t$ where here $t$ plays the role played by $k$ in part (b) [it is computed analogously to there].
The prover then sends back a string $x_1 \cdot x_2 \dots x_{M} \in \{0, 1\}^{MR}$ for some multiplier $M$ (called $n$ in part (b)), and an integer $i$, claiming that (1) $h_i(x_1, \dots, x_M) = y$ and (2) each $x_j$ is in the set $S$.
The verifier can immediately verify condition (1).
To enable the verifier to confirm condition (2), the prover simply needs to send the verifier the messages it would have sent in a round of interaction where the verifier sent the string $x_j$; the verifier can then check that it would have accepted on this response.

\medskip
\noindent
\textbf{Analysis.}
Per the analysis in part (b), this procedure has perfect completeness.
To show soundness all that we need to do is ensure that the verifier can succeed
in checking whether $x_j \in S$.
Well, say $x_j \notin S$.
Due the assumed optimality of $P$, this implies there does not exist a response to string $x_j$ which would have caused the verifier to accept.
Thus, the prover in the protocol in the previous paragraph could not have produced a message to convince the verifier $x_j \in S$.
(This shows that in the protocol above, the verifier never mistakenly thinks that a string $x_j \in S$.
It does not show the protocol has perfect soundness, for the same reason as in part (b): there
is a small but nonzero chance that the verifier happens to choose a string $y$ where some sequence $x_1 \dots x_M$ of values in $S$ hash to $y$ under one of the $h_i$.)

\newpage
\section*{Problem 5: The Limits of PCPs (4 Points, 2 for each sub-problem)}
\subsection*{Question}
Recall that in class we defined $\mf{PCP}_s[r(n),q(n)]$ to be the set of functions with probabilistically checkable proofs having ``soundness'' $s$. In general, we can parametrize the ``completeness'' as well. 

Specifically, define $f:\{0,1\}^* \rightarrow \{0,1\}$ to be in $\mf{PCP}_{c,s}[r(n),q(n)]$ if there is a probabilistic polynomial time algorithm $V$ such that for all $x$, $V$ uses $O(r(|x|))$ random bits, asks $q(|x|)$ oracle queries to a proof string $P$ non-adaptively, must decide whether accept or reject, and
\begin{itemize}
	\item
	$f(x) = 1 \Longrightarrow$ there is a $P$ such that $\Pr[V^P(x) \textrm{ accepts}] \geq c$.
	\item
	$f(x) = 0 \Longrightarrow$ for all $P$, $\Pr[V^P(x) \textrm{ accepts}] < s$.
\end{itemize}

Note that in this generalized version, when $f(x) = 1$, we do not require the verifier to accept with probability $1$ on some proof $P$. 

In the PCP lectures, it was proved that $\mf{PCP}_{1,1}(\log n, 3)=\mathbf{NP}$. The number $3$ here is actually the smallest possible. In this problem, you are asked to show that if we reduce the number of queries to two or one, the classes become $\mathbf{P}$. Prove that:
\begin{enumerate}[(a)]
	\item
	for every $0<s\leq  c\leq 1$, $\mf{PCP}_{c,s}(\log n,1)=\mathbf{P}$.
	\item
	for every $0<s\leq 1$, $\mf{PCP}_{1,s}(\log n,2)=\mathbf{P}$.
\end{enumerate} 
\emph{Hint: Think about these 1-query and 2-query PCPs from the CSP/inapproximability perspective: what you want to show is that the resulting CSPs are in fact easy to solve.}

\smallskip

{\bf Extra credit:} Prove that for every $0<s\leq 1$, $\bigcup_{k \geq 1} \mf{PCP}_{1,s}(n^k,2) \subseteq \mf{PSPACE}$

\smallskip

\emph{Hint: Use the fact that 2SAT is in NL.} 

\newpage
\subsection*{Answer to (a)}
For contradiction, say $\mathbf{PCP}_{c, s}(\log n, 1) \not\subseteq \P$.
Let $f \in \mathbf{PCP}_{c, s}(\log n, 1) \setminus \P$ be a decision problem,
decided by verifier $V$.
I will contradict this by constructing a polynomial time Turing machine that decides $f$.
It will operate as follows.  Say input $x$ is given.
For each $r \in \{0, 1\}^{\log n}$, let $y_r = 1$ if $V(x, r, 1) = 1$ and let $y_r = 0$ otherwise. (Each of these can be computed in polytime.) Here I write $V(x, r, y)$ to denote the value returned by verifier $V$ on input $x$ and randomness $r$, given that it received value $y$ after the one query it makes to the prover (which is a deterministic function of $x$ and $r$).

Observe that if $f(x) = 1$, then $V(x, r, y_r) = 1$ for at least $c$ of the possible $r$ values.
If $f(x) = 0$, then $V(x, r, y_r) = 0$ for at least $1 - s$ of the possible $r$ values (and in fact this would be true for any assignment to the $y$ values).
Since there are only $n$ distinct $y_r$ values, our algorithm can simply compute $V(x, r, y_r)$ for each $r \in \{0, 1\}^n$, and accept if $\geq cn$ of them are true.

\subsection*{Answer to (b)}
In this problem I will write $V(x, r, y, z)$ to be the value that $V$ outputs on input $x$ and randomness $r$, given that it receives response $y$ and $z$ from the prover on the two queries it makes.  Since the queries are non-adaptive, for any verifier, $y$ and $z$ are a function of $x$ and $r$.

Again for contradiction, say $\mathbf{PCP}_{1, s}(\log n, 1) \not\subseteq \P$.
Let $f \in \mathbf{PCP}_{1, s}(\log n, 1) \setminus \P$.
I will describe a polytime algorithm that decides $f$.

Fix input string $x$.
We will construct a 2SAT instance on variables $Y_0, \dots, Y_{n-1}, Z_0, \dots, Z_{n-1}$.
For each $r \in \{0, 1\}^{\log n} = \{0, 1, \dots, n-1\}$, we will construct a clause
$C_r$ as follows.
Run $V(x, r, y, z)$ on this $x$ and $r$ for each of the 4 permutations of possible
values for $y, z$.
Let $C_r$ be a 2SAT formula on variables $Y_r, Z_r$ so that $C_r(y, z) = V(x, r, y, z)$ for all $y$, $z$.\footnote{
Here are the details of the construction of $C_r$.
If $V(x, r, y, z)$ is never 1, set $C_r = Y_r \wedge \neg Y_r$.
If $V(x, r, y, z) = 1$ only on $y^*, z^*$, set $C_r = (Y_r = y^*) \wedge (Z_r = z^*)$.
If $V(x, r, y, z) = 1$ exactly when $y = 1$, set $C_r = Y_r$; do the analogus thing if it is 1 exactly when $y = 0$ or exactly when $z = 1$ or exatly when $z = 0$.
If $V(x, r, y, z) = 1$ exactly when $y = z$ make $C_r = (Y_r \vee \neg Z_r) \wedge (\neg Y_r \vee Z_r)$.
If $V(x, r, y, z) = 1$ exactly when $y \neq z$ make $C_r = (Y_r \vee Z_r) \wedge (\neg Y_r \vee \neg Z_r)$.
If $V(x, r, y, z) = 1$ in every case except $(y^*, z^*)$, set $C_r = (Y_r \neq y^*) \vee (Z_r \neq z^*)4$.
If $V(x, r, y, z) = 1$ on all $y, z$, set $C_r = \eps$, the empty clause.
% 	If $V(x, r, y, z) = 1$ for 3 assignments to $y$ and $z$, let $C_r = (\neg) Y_r \vee (\neg) Z_r$ where we optionally include negations $\neg$, to make $C_r$ the clause which accepts exactly the 3 permutations of $(y, z)$ that $V$ accepted.
% 	If $V(x, r, y, z) = 1$ for 2 assignments)
}

If $f(x) = 1$, then for every $r$, there is an assignment to $Y_r, Z_r$ so that $C_r$ is satisfied (this is because $c = 1$).
If $f(x) = 0$, then for every assignment to $Y_r, Z_r$,
$< sn$ of the $C_r$ are satisfied.
Let $C = \wedge_{r} C_r$, which is a conjunction of 2CNF formulae, and is thus a 2CNF formula itself.
We can determine whether $C$ is satisfiable in polynomial time (this is just a 2SAT problem); if it is, then we must have $f(x) = 1$, and if not, we must have $f(x) = 0$.

\newpage
\section*{Acknowledgements}
I went to Zixuan's office hours, and she helped me understand how to solve problems 2 and 4.

\end{document}